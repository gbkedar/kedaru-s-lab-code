\documentclass[12pt]{report}
\usepackage{graphicx}
\usepackage{verbatim}
\usepackage{amsmath}
\usepackage{amsthm}
\usepackage{amssymb}

\providecommand{\norm}[1]{\lVert#1\rVert}

\begin{document}

\begin{center}
	\large{Optimization Theory - MATH 6366}\\
	\hfill \hfill \large{Homework \#2} \hfill \large{Kedar Grama}\\
\end{center}

\section*{Problem 1:}
Let $$f (x) = x^4 + a_1 x^3 + a_2 x^2 .$$
(i) For what values of $\{a_1 , a_2 \}$ is $f$ convex on $\mathbb{R}$?\\
$Solution:$ Using the second order sufficiency condition on p.71 in the reference book
by Boyd $et\, al.$, for a continuously differentiable function in $\mathbb{R}$ we only need
$f'' ≥ 0$. In this case we have:
\begin{align*}
f (x) &= x^4 + a_1 x^3 + a_2 x^2\\
f'(x) &= 4x^3 + 3a_1 x^2 + 2a_2 x\\
f''(x) &= 12x^2 + 6a_1 x + 2a_2 > 0 \, , x \in \mathbb{R}
\end{align*}
So the choices are $a_1=0$ and $a_2\geq0$.\\
\\
(ii) For what values of $\{a_1 , a_2 \}$ is $f$ convex on $[0, \infty)$?\\
$Solution:$ Again, we need:
$$f''(x) = 12x^2 + 6a_1 x + 2a_2 > 0 \, , x \in \mathbb{R}^+$$
So the choices are $a_1\geq0$ and $a_2\geq0$.\\
\\
(iii) Suppose $g(x) := x^4 - c x^2$ with $c > 0$.\\
(a) Find the critical points of $g$ and indicate which critical points are local
minimizers.\\
$Solution:$ The solutions to $g'(x) = 2x(2x^2 - c )=0$  are the critical points of $g(x)$.
They are $x=0$ and $x= \pm \sqrt{\frac{c}{2}} $. The second derivative is\\
$g''(x) = 12x^2 - 2c $. $g''(\pm \sqrt{\frac{c}{2}})=6c-2c$, so the points
$x=\pm \sqrt{\frac{c}{2}}$ are local minimizers if $c>0$. $g''(0)=-2c$, so the point $x=0$
is a local maximizer since $c>0$.\\
\\
(b) Find $\alpha(c) := inf_{x \in \mathbb{R}} g(x)$ and the points where this value is
attained.\\
$Solution:$ $\alpha(c) = inf_{x \in \mathbb{R}} g(x) = \frac{c^2}{4} - \frac{c^2}{2}=
\frac{-c^2}{4}$ The points where they attained are at $x= \pm \sqrt{\frac{c}{2}} $.\\
\\
(c) On what interval(s) is this function convex?\\
$Solution:$ It is easy to see that $g(x)=g(-x)$, that is, the function is symmetric about the
$x-$axis. Since $x=0$ is a local maximizer and the only critical points outside $x=0$ are
local minimizers, the function is convex on $[0,\infty)$ and $(-\infty,0]$ \\

\section*{Problem 2:}
Consider the following functions on $(0, \infty)$
$$f_1 (x) := \sqrt{x} \ln x, \, f_2 (x) := x \ln x, \, f_3 (x) := x^2 \ln x,
\, f_4 (x) := x \ln (1 + x)$$
(i) Which of these functions are increasing on $\mathbb{R}^+$?\\
$ f_1'(x) = \frac{1}{\sqrt{x}} (\ln x+1) $ is not always greater than 0 in $x\in (0,1)$ so it
is not always increasing\\
$ f_2'(x) = \ln x+1 $ is not always greater than 0 in $x\in (0,1)$ so it is not always
increasing\\
$ f_3'(x) = x(2\ln x + 1)$ is not always greater than 0 in $x\in (0,1)$ so it is not always
increasing\\
$ f_4'(x) = \frac{x}{(1 + x)} + \ln(1 + x) $ is greater than 0 in $x\in \mathbb{R}^+$ so it is
always increasing\\
\\
(ii) Find $\lim_{x\rightarrow 0+} f_j (x)$ for each $j \in \{1, 2, 3, 4\}$.\\
Writing $\ln x = (x-1) - \frac{(x-1)^2}{2} + \frac{(x-1)^3}{3} - \frac{(x-1)^4}{4} + .. $\\
$\lim_{x\rightarrow 0+} f_1 (x) = \lim_{x\rightarrow 0+} \sqrt{x} \left(
(x-1) - \frac{(x-1)^2}{2} + \frac{(x-1)^3}{3} - \frac{(x-1)^4}{4} + ..\right) = 0 $\\
$\lim_{x\rightarrow 0+} f_2 (x) = \lim_{x\rightarrow 0+} x \left(
(x-1) - \frac{(x-1)^2}{2} + \frac{(x-1)^3}{3} - \frac{(x-1)^4}{4} + ..\right) = 0 $\\
$\lim_{x\rightarrow 0+} f_3 (x) = \lim_{x\rightarrow 0+} x^2 \left(
(x-1) - \frac{(x-1)^2}{2} + \frac{(x-1)^3}{3} - \frac{(x-1)^4}{4} + ..\right) = 0 $\\
$\lim_{x\rightarrow 0+} f_4 (x) = \lim_{x\rightarrow 0+} x \left(
x - \frac{x^2}{2} + \frac{x^3}{3} - \frac{x^4}{4} + ..\right) = 0 $\\
The above is true as all the series are convergent to a finite number.\\
\\
(iii) Which of these functions is convex on $[0, \infty)$?\\
$ f_1''(x) = \frac{-1}{2x^{\frac{3}{2}}} (\ln x+1) + \frac{1}{x^{\frac{3}{2}}} $ is not always
greater or equal to 0, hence non-convex in $[0, \infty)$\\
$ f_2''(x) = \frac{1}{x}>0$, hence convex in $[0, \infty)$\\
$ f_3''(x) = \ln x + 3 $ is not always greater or equal to 0, hence non-convex in $[0, \infty)$\\
$ f_4''(x) = \frac{x}{(1 + x)^2} + \frac{1}{x}>0 $, hence convex in $[0, \infty)$\\
\\
(iv) Find $\inf_{x>0} f_j (x)$ for each $j \in \{1, 2, 3, 4\}$. When these quantities are
finite, find the minimizers of each of these functions.\\
The minimizers are:\\
$ f_1'(x) = \frac{1}{\sqrt{x}} (\ln x+1) = 0$ or $\ln x=-1$. $x=e^{-1}$ is the minimizer for
$ f_1(x)$\\
$ f_2'(x) = \ln x+1 = 0$ or $\ln x=-1$. $x=e^{-1}$ is the minimizer for $ f_2(x)$\\
$ f_3'(x) = x(\ln x + 1) = 0 $ or $\ln x=-1$.  $x=e^{-1}$ is the minimizer for $ f_3(x)$\\
$ f_4'(x) = \frac{x}{(1 + x)} + \ln(1 + x) = 0$ or $x=0$. $x=0$ is the minimizer for $ f_4(x)$\\
\\
\section*{Problem 3:}
Let C be a nontrivial convex set in $\mathbb{R}^N$ and $f_1$, $f_2$ be continuous convex real
valued functions on C.\\
(i) Prove that $g(x):=\max\{f_1(x), f_2(x)\}$ is a continuous convex function on C.\\
For convexity we need to prove
\begin{equation}
g(tx_1+(1-t)x_2)\leq t g(x_1)+(1-t)g(x_2) \quad ,t\in(0,1), x_1\neq x_2
\end{equation}
Let us consider the case when $f_1(x_1) >f_2(x_1)$ and $f_2(x_2) >f_1(x_2)$. For any
$g(tx_1+(1-t)x_2)\in f_1$ we have $g(tx_1+(1-t)x_2)\leq t f_1(x_1)+(1-t)f_1(x_2)$ and for \\
$g(tx_1+(1-t)x_2)\in f_2$ we have $g(tx_1+(1-t)x_2)\leq t f_2(x_1)+(1-t)f_2(x_2)$. The bound in
(1) is looser than the above two as $f_1(x_1) >f_2(x_1)$ and $f_2(x_2) >f_1(x_2)$ hence it holds.
Similarly we can show that the function is convex in the other cases too. \qed
\\
\\
(ii) Let C = $\mathbb{R}$, $f_1(x):=x-1$, $f_2(x):=mx+1$ and $g$ is defined as in part(i). Find
$$\alpha(m):=\inf_{x\in \mathbb{R}}g(x).$$\\
The two lines $f_1(x):=x-1$ and $f_2(x):=mx+1$ intersect when $m\neq1$. If
$m=1$, $\inf_{x\in \mathbb{R}}g(x)=-\infty$. When $m\neq1$ it is easy to see the minimizer is the
intersection of the two lines which occurs at $x=\frac{2}{1-m}$ \\
\\
(iii) Show that if $g$ in (ii) has a unique minimizer $x$, then $g$ is not differentiable at
$x$.\\
$g$ has a unique minimizer as shown above when the two lines intersect and the minimizer is at
$x=\frac{2}{1-m}$. At $x=\left(\frac{2}{1-m}\right)^+$ and $x=\left(\frac{2}{1-m}\right)^-$ the
derivatives are 1 and $m$ the order of the derivatives depends on the sign of $m$. Thus it is not
differentiable at the minimizer.
\end{document}
