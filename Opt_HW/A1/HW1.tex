\documentclass[12pt]{report}

\usepackage{amsmath}
\usepackage{amsthm}
\usepackage{amssymb}
\usepackage{mcode}

\begin{document}

\begin{center}
	\large{MATH-6366}\\
	\hfill \hfill \hfill \large{Homework \#1} \hfill \large{Kedar Grama}\\
\end{center}

\section*{Problem 1:}Consider the polynomial $P (x, b) := 2 x^6 + 3 x^4 - 6 b x^2 - 12 x$ with $b$ a real parameter.\\
(i) Show that $P( . , 0)$ has a unique local minimizer on $ \mathbb{R} $ and give the equation obeyed by a local minimizer of $P (x, 0)$.\\
(ii) Show that this minimizer occurs in $(0, 1)$ and is a global minimizer of $P (., 0)$.

$Solution:$ It is easy to see that the function is continuously differentiable as it is a polynomial. The first derivative of the
function $P(x,0) = 2 x^6 + 3 x^4 - 12 x$ is $P'(x,0)=12 x^5 + 12 x^3 - 12$. A critical point follows the equation
$12 x^5 + 12 x^3 - 12 = 0 $ or 
\begin{equation}
x^5 + x^3 - 1 = 0
\end{equation}

From the second order sufficiency condition on p.71 in the reference book by Boyd et al., to prove convexity for a
continuously differentiable function in $\mathbb{R}$ we only need to prove $P''(x,0)\geq0$, $P''(x,0) = 60x^4+36x^2 \geq 0$ as we
only have even powers of the variable. Hence, the function is convex.

We know that a critical point satisfies the equation $x^5+x^3=1$. Consider the following cases:\\
(a) $x\leq 0$: At $x=0$, $x^3+x^5 = 0 \neq 1$, for $x<0$, $x^3+x^5<0 \neq 1$\\
(b) $x\geq 1$: At $x=1$, $x^3+x^5 = 2 \neq 1$, for $x>1$, $x^3+x^5>2 \neq 1$\\
Hence, all the critical points are in the interval $(0,1)$. But, we know that the function is strictly convex in the same interval as
$P''(x,0)=60x^4+36x^2>0, x \in (0,1)$.

Using the above statements we can say that the function is convex, critical points lie in the interval $(0,1)$ and also that the
function is strictly convex in the same interval. Thus we can say that there is only one minimizer and follows (1).
\\
\\
(iii) Find an interval of length $\delta \leq 0.01$ that contains the minimizer, include a description of how you found this
interval.\\
$Solution:$ I used the Matlab code included in the next page and found that the optimized lies in the interval (0.8369,0.8379):
\begin{lstlisting}
%Starting points
b_j = 1;
a_j = 0;
z_j = 0.5;
%Initializations
tolReached = 1;
%Search loop
while( tolReached )
    z_j = 0.5*(a_j+b_j);
    d_j = EvalFnDer(z_j);
    if( abs(d_j)<10*eps )
        tolReached = 0;
    end
    if d_j>0
        b_j = z_j;
    else
        a_j = z_j;
    end
    if abs(a_j-b_j)<0.001
        tolReached = 0;
    end
end
a_j
b_j
EvalFn(0.5*(a_j+b_j))
function outVal = EvalFnDer( x )
    outVal = 12*x.^5 +12*x.^3-12;
end
function outVal = EvalFn( x )
    outVal = 2*x.^6 +3*x.^4-12*x;
end
\end{lstlisting}

(iv) Give (good) lower and upper bounds on the value of this problem.\\
$Solution:$ The function values lie in the interval $(-7.8839,\infty)$

\pagebreak

\section*{Problem 2:}
(i) Show that for any $b\in\mathbb{R}$, the value of $$\alpha(b):=\inf_{x\in\mathbb{R}}P(x,b) \,\,  \mathrm{is\,finite}$$\\
$Solution:$ Let us consider two cases. First, when $b\in(-\infty,1]$ then for any $|x|>2$ the leading terms are $2x^6 + 3x^4$ and
the $\inf P(.,b)$ always occurs in $x\in (-2,2)$. Since $b$ is finite the and the rest of the terms belong to a bounded polynomial
the $\inf_{x \in \mathbb{R}}$ is bounded. Second, when $b\in (1,\infty )$, the leading term for the $\inf$ is $-6bx^2$ but
for any $|x|\approx b$ the first two terms $2x^6 + 3x^4$ are larger than $-6bx^2$. So, the $\inf$ always occurs when $|x|<<b$
hence in $-6bx^2$, $b$ is finite and $|x|<<b$ so the $\inf_{x \in \mathbb{R}}$ is bounded.
\\
\\
(ii) Show that $P(.,b)$ cannot have more than 3 critical points for any value of $b\in\mathbb{R}$, and has only one critical
point when $b<0$.\\
$Solution:$ We observe:
\begin{align*}
P(.,b) &= 2x^6+3x^4-6bx^2-12x \\
P'(.,b) &= 12(x^5+x^3-bx-1) \\
P''(.,b) &= 12(5x^4+3x^2-b) \\
P'''(.,b) &= 12(20x^3+6x) = 12x(20x^2+6)\\
\end{align*}
Also $x=0$ is not a root of $P'(.,b)$ and hence is not a critical point. $P'''(.,b)$ has only one real root at $x=0$
and it's derivative is $P''''(.,b)=72(10x^2+1)>0$, strictly positive hence we can say that $P''(.,b)$ is strictly convex. $P''(.,b)$
has a value of $P''(0,b)=-12b$ and it is the minimum and the function is symmetric on both sides of $x=0$.  Also, this is a
monotonically increasing function. Any integral of this function can at most two critical points and hence at most three zero
crossings. Hence there are at most 3 critical points in $P(.,b)$.

Since, $P''(.,b) = 12(5x^4+3x^2-b)>0, x\in \mathbb{R}, b<0$, and the earlier observation that there were no critical points at
$x=0$ we can say that the function $P(.,b), b<0$ is strictly convex. Hence it has only one critical point.
\\
\\
(iii) Show that, for any value of $b$, $P(.,b)$ has only one critical point $\hat{x}(b)$ in $(0,\infty)$ and that this critical
point is a local minimizer of $P (., b)$.\\
$Solution:$ From (ii) $P'''(., b)>0, x\in (0,\infty)$. Hence, $P'(.,b),x\in (0,\infty)$ is a convex function with a value of
$P'(0^+,b)\approx-12$ and it is monotonically increasing as $P'(0,b)\to \infty$ as $x\to \infty$ thus the first derivative must be
zero at only one value. Hence there is only one critical point of $P(.,b)$ in $x\in (0,\infty)$. To show that this is a local
minimum, solving for $b$ at $P'(., b)=0$ we get $b=12x^4+12x^2-\frac{1}{x}$. Substituting for $b$ in $P''(., b)$ we get
$P''(., b)=12(4x^4+2x^2+\frac{1}{x}) > 0 $ when $P'(., b)=0, x \in (0,\infty)$. Hence we have proved that it is a local minimum.
\\
\\
(iv) Show that for large enough $b$, $P (., b)$ has three critical points.\\
$Solution:$ For $P'(.,b) = 12(x^5+x^3-bx-1), b>>0$, when $x<0, |x| \geq b$, it is easy to see that $|x^5+x^3-1|>|-bx|$. Hence for a
large enough $b$, $P(.,b)<0$ in $x\in (-\infty,0)$. There is also an $x<0, |x|<b$ where $|x^5+x^3-1|<|-bx|$ so $P(.,b)>0$ and we
know $P'(.,b) = -12$. Thus we have shown that there are two zero crossings of the first derivative hence there are two critical
points and we have already shown that there is one and only one local minima in $x>0$.
\\
\\
(v) Prove that $\hat{x}(b)$ is always the global minimizer of $P (., b)$.\\
$Solution:$ We have already shown for $b\leq 0$ that the function is convex. Hence $\hat{x}(b)$ is a global minimizer. In the case where $b>0$ we have for any $x_0<0$ there exists an $x_1=|x_0|$, $P(.,b) = 2x_0^6+3x_0^4-6bx_0^2-12x_0 > 2x_1^6+3x_1^4-6bx_1^2-12x_1 $ as $b$ is a positive constant. Hence $\hat{x}(b)$ is always a global minimizer.
\\
\\(vi) Find the equations for the value $b_c$ of $b$ with the property that the function $P (., b_c )$ has exactly two critical
 points. Can you solve these equations (numerically or otherwise)?
\\
$Solution:$ $b_c$ is the solution to this optimization problem:
\begin{align*}
&\max_{b\in (0,\infty )} \, b\\
b\leq & \frac{x^5+x^3-1}{x}, x \in (-\infty,0)
\end{align*}

\end{document}

