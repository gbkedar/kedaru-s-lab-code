\documentclass[12pt]{report}

\usepackage{amsmath}
\usepackage{amsthm}
\usepackage{amssymb}
\usepackage{mcode}

\begin{document}

\begin{center}
	\large{MATH-6366}\\
	\hfill \hfill \hfill \large{Homework \#1} \hfill \large{Kedar Grama}\\
\end{center}

\section*{Problem 1:}Consider the polynomial $P (x, b) := 2 x^6 + 3 x^4 - 6 b x^2 - 12 x$ with $b$ a real parameter.\\
(i) Show that $P( . , 0)$ has a unique local minimizer on $ \mathbb{R} $ and give the equation obeyed by a local minimizer of $P (x, 0)$.\\
(ii) Show that this minimizer occurs in $(0, 1)$ and is a global minimizer of $P (., 0)$.

$Solution:$ It is easy to see that the function is continuously differentiable as it is a polynomial. The first derivative of the
function $P(x,0) = 2 x^6 + 3 x^4 - 12 x$ is $P'(x,0)=12 x^5 + 12 x^3 - 12$. A critical point follows the equation
$12 x^5 + 12 x^3 - 12 = 0 $ or 
\begin{equation}
x^5 + x^3 - 1 = 0
\end{equation}

From the second order sufficiency condition on p.71 in the reference book by Boyd et al., to prove convexity for a
continuously differentiable function in $\mathbb{R}$ we only need to prove $P''(x,0)\geq0$, $P''(x,0) = 60x^4+36x^2 \geq 0$ as we
only have even powers of the variable. Hence, the function is convex.

We know that a critical point satisfies the equation $x^5+x^3=1$. Consider the following cases:\\
(a) $x\leq 0$: At $x=0$, $x^3+x^5 = 0 \neq 1$, for $x<0$, $x^3+x^5<0 \neq 1$\\
(b) $x\geq 1$: At $x=1$, $x^3+x^5 = 2 \neq 1$, for $x>1$, $x^3+x^5>2 \neq 1$\\
Hence, all the critical points are in the interval $(0,1)$. But, we know that the function is strictly convex in the same interval as
$P''(x,0)=60x^4+36x^2>0, x \in (0,1)$.

Using the above statements we can say that the function is convex, critical points lie in the interval $(0,1)$ and also that the
function is strictly convex in the same interval. Thus we can say that there is only one minimizer and follows (1).
\\
\\
(iii) Find an interval of length $\delta \leq 0.01$ that contains the minimizer, include a description of how you found this
interval.\\
$Solution:$ I used the Matlab code included in the next page and found that the optimized lies in the interval (0.8369,0.8379):
\begin{lstlisting}
%Starting points
b_j = 1;
a_j = 0;
z_j = 0.5;
%Initializations
tolReached = 1;
%Search loop
while( tolReached )
    z_j = 0.5*(a_j+b_j);
    d_j = EvalFnDer(z_j);
    if( abs(d_j)<10*eps )
        tolReached = 0;
    end
    if d_j>0
        b_j = z_j;
    else
        a_j = z_j;
    end
    if abs(a_j-b_j)<0.001
        tolReached = 0;
    end
end
a_j
b_j
EvalFn(0.5*(a_j+b_j))
function outVal = EvalFnDer( x )
    outVal = 12*x.^5 +12*x.^3-12;
end
function outVal = EvalFn( x )
    outVal = 2*x.^6 +3*x.^4-12*x;
end
\end{lstlisting}

(iv) Give (good) lower and upper bounds on the value of this problem.\\
$Solution:$ The function values lie in the interval $(-7.8839,\infty)$

\pagebreak

\section*{Problem 2:}
(i) Show that for any $b\in\mathbb{R}$, the value of $$\alpha(b):=\inf_{x\in\mathbb{R}}P(x,b) \,\,  \mathrm{is\,finite}$$\\
\\
(ii) Show that $P(.,b)$ cannot have more than 3 critical points for any value of $b\in\mathbb{R}$, and has only one critical
point when $b<0$.\\
\\
(iii) Show that, for any value of $b$, $P(.,b)$ has only one critical point $\hat{x}(b)$ in $(0,\infty)$ and that this critical
point is a local minimizer of $P (., b)$.\\
\\
(iv) Show that for large enough $b$, $P (., b)$ has three critical points.\\
\\
(vi) Find the equations for the value $b_c$ of $b$ with the property that the function $P (., b_c )$ has exactly two critical points. Can you solve these equations (numerically or otherwise)?
