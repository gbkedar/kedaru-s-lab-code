\documentclass[12pt]{report}
\usepackage{graphicx}
\usepackage{verbatim}
\usepackage{amsmath}
\usepackage{amsthm}
\usepackage{amssymb}

\providecommand{\norm}[1]{\lVert#1\rVert}

\begin{document}

\begin{center}
	\large{Optimization - MATH 6366}\\
	\hfill \hfill \large{Homework \#4} \hfill \large{Kedar Grama}\\
\end{center}

\section*{Problem 1:}
Suppose that $f:\mathbb{R}^2\rightarrow\mathbb{R}$ is continuously differentiable and
has a local minimizer on the set $\mathbb{R}^2_+:=[0,\infty)\times [0,\infty)$ at a
point $(a,0)$ with $a>0$.\\
(i) What extremality conditions hold at this point? \\
(ii) When $(0, 0)$ is a local minimizer of $f$ on $\mathbb{R}^2_+$ what conditions must
hold at $(0, 0)$?\\
(i) The extremality condition that holds at this point is:
$$ \langle \nabla f(a,0), y-(a,0)^T \rangle \geq 0 \quad \forall y \in \mathbb{R}^2_+ $$
(ii) In this case, as it is an edge point, there is one inequality:
$$\langle \nabla f(0,0), y \rangle  \geq 0 \quad \forall y \in \mathbb{R}^2_+$$
or
$$ \nabla f(0,0) \in \mathbb{R}^2_+ $$

\section*{Problem 2:}
Let $\Delta'_N$ be the set of all probability vectors in $\mathbb{R}^N$ and
$f:\mathbb{R}^N\rightarrow\mathbb{R}$ be a continuously differentiable function.\\
(i)What extremality conditions hold at $e^{(1)}$ when $e^{(1)}=(1,0,\dots,0)$ is a
local minimizer of $f$ on $\Delta'_N$?\\
(ii)Suppose that $f$ is maximized on $\Delta'_N$ at a point a with each $a_j>0$.
What extremality conditions hold at the point $a$?\\
(iii) Find the maximum and minimum values of $f(x):=x_1x_2\dots x_N$ on $\Delta'_N$.\\
(iv) Use the preceding result to find an upper bound for $f(x)$ that holds for all
$x \in \mathbb{R}^N_+$ and involves $\sum_{j=1}^N x_j$\\
\\
$Solution:$ (i) $\Delta'_N:=\{ x \in \mathbb{R}^N: \sum_{i=1}^N x_i=1, x_i\geq 0 \}
\implies e^{(1)}$ is an edge point of $\Delta'_N$ the extremality conditions that hold
at $e^{(1)}=(1,0,\dots,0)$ when it is a minimizer of $f$ on $\Delta'_N$ is that
$$\nabla f(e^{(1)}) \in N_{e^{(1)}}(\Delta'_N)$$ 
Let $\nabla f(e^{(1)})=d$, from the above we have:
\begin{align*}
d \in \mathbb{R}^N: \langle d, y-e^{(1)} \rangle &\geq 0, \quad y \in \Delta'_N \\
-d_1 + \sum_{i=1}^N d_i y_i \geq 0, \quad \sum_{i=1}^N y_i =1, y_i \geq 0 \forall i \\
\sum_{i=1}^N d_i y_i \geq d_1, \quad \sum_{i=1}^N y_i =1, y_i \geq 0 \forall i 
\end{align*}
$\implies d_1\leq d_2, d_1\leq d_3, \dots ,d_1\leq d_N$
\\
(ii) Let us call the point $\hat{x}$ and $\nabla f(\hat{x}):=d$. Using the same
notation as the above:
$$\sum_{i=1}^Nd_iy_i\leq\sum_{i=1}^N d_i a_i ,\quad\sum_{i=1}^N y_i =1,y_i\geq 0
\text{ and } 0<a_i<1 \forall i $$
$\implies d_1 = d_2 = \dots = d_N = 0$ or $\nabla f(\hat{x})=0$
\\
(iii) When $f(x):=x_1x_2 \dots x_N$ on $\Delta'_N$ since $x_j\geq 0$, the minimum
value of $f(x)$ is 0 and for the maximum, as
$\sqrt[N]{x_1x_2\dots x_N}\leq(\frac{x_1+x_2+\dots x_N}{N})^N=(\frac{1}{N})^N$. Hence
the maximum value of $f(x)$ is $(\frac{1}{N})^N$ and is attained when
$x_1=x_2=\dots =x_N=\frac{1}{N}$.\\
\\
(iv) When $f(x)$ is defined on $\mathbb{R}^N_+$, similarly we have
$f(x)\leq \frac{(\sum_{i=0}^N x_i)^N}{N^N}$ and again this is attained when
$x_1=x_2=\dots =x_N$. So the upper bound is
$\frac{(\sum_{i=0}^N x_i)^N}{N^N} \in [0,+\infty)$

\section*{Problem 3:}
Suppose $y \in \mathbb{R}^N$ and $K$ is the hyperplane defined by
$\langle a,x \rangle = b$ with $\|a\|_2 = 1$. Show that the Euclidean distance from
$y$ to $K$ is $d(y,K) = |\langle a,y \rangle - b|$ by solving a minimization
problem.\\
Interpret this result geometrically when $N=2$, $a=\frac{1}{\sqrt{2}}(1,1)$ and find
the closest point on the line $K$ to $y = (2, 3)$ when $b = 2^{-\frac{1}{2}}$. \\
\\
$Solution:$ Let $f(x):=\|x-y\|_2^2$ constrained by
$K:=\{x\in\mathbb{R}^N,\langle a,b \rangle = b\}$. Let $\alpha:= \inf_{x\in K} f(x) $
and $g(x):=\langle a,b \rangle -b$. So, $d(y,K)=\sqrt{\alpha}$. By the Lagrange
multiplier rule, if $\hat{x}$ minimizes $f(x)$ on $K$, $\exists$ a $\lambda \in 
\mathbb{R}_+$ such that $\nabla f(\hat{x})=\lambda \nabla g(\hat{x})$ and 
$ g(\hat{x})=0$. Hence, by multiplying the following by $a^T$ and $\alpha$
\begin{align*}
\alpha(\hat{x}-y) &=\lambda a \quad & \quad \langle \hat{x},a\rangle&=b\\
\alpha a^T\hat{x}-\alpha a^T y &= a^Ta \lambda =  \|a\|_2^2 \lambda = \lambda \quad &
\quad \alpha a^T \hat{x} &= \alpha b
\end{align*}
Using the above two equations, we have $\lambda = \alpha(b-a^Ty)$ and 
$\hat{x} = y+\frac{1}{\alpha} \lambda a$. Thus
$\alpha = \|(y+\frac{\lambda a}{\alpha})-y\|_2^2 = \|\frac{\lambda a}{\alpha}\|^2_2
\implies d(y,K)= \|\frac{\lambda a}{\alpha}\|_2=|\frac{\lambda}{\alpha}|\|a\|_2 
= |\frac{\lambda}{\alpha}|$ or $d(y,K)= |\langle y,a \rangle-b|$

When  $N=2$, $a=\frac{1}{\sqrt{2}}(1,1)$, $y = (2, 3)$ and $b = 2^{-\frac{1}{2}}$ the
given formulation solves for the distance between the point $y$ and the line
$x_1+x_2=1$ and from the above result 
$d=|2^{-\frac{1}{2}}(2+3)-2^{-\frac{1}{2}}|=2\sqrt 2$ and the closest point 
$y^* = y-da = (2,3)-(\frac{2\sqrt 2}{\sqrt 2},\frac{2\sqrt 2}{\sqrt 2})=(0,1)$

\end{document}

