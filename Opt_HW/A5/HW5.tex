\documentclass[12pt]{report}
\usepackage{graphicx}
\usepackage{verbatim}
\usepackage{amsmath}
\usepackage{amsthm}
\usepackage{amssymb}

\providecommand{\norm}[1]{\lVert#1\rVert}

\begin{document}

\begin{center}
	\large{Optimization - MATH 6366}\\
	\hfill \hfill \large{Homework \#5} \hfill \large{Kedar Grama}\\
\end{center}

\section*{Problem 1:}
The matrices are:
\begin{align*}
A &= \left(\begin{array}{c c}
1 & 1\\
2 & 4\\
1 & -1
\end{array} \right),
A^T = \left(\begin{array}{c c c}
1 & 2 & 1\\
1 & 4 & -1
\end{array} \right),
A^TA = \left(\begin{array}{c c}
6 & 8\\
8 & 18\\
\end{array} \right),
b = \left(\begin{array}{c c}
1\\
3\\
1
\end{array} \right)\\
A^Tb&= \left(\begin{array}{c c}
8\\
12
\end{array} \right)\\
\end{align*}
The sketch is:\\
\includegraphics[scale=0.5]{fig1}

\section*{Problem 2:}
The points of intersection are:
$L1,L2 = p_1(\frac{1}{2},\frac{1}{2})$,
$L2,L3 = p_2(\frac{7}{6},\frac{1}{6})$, and \\
$L3,L1 = p_3(1,0)$\\
\\
The parametric a for points inside the triangle with these three points as vertices is the
Barycentric coordinate system which is given by:\\
$\textbf{r} = \lambda_{1} \textbf{p}_{1} + \lambda_{2} \textbf{p}_{2} + \lambda_{3}
\textbf{p}_{3} $
where $\lambda_{1}$, $\lambda_{2}$, $\lambda_{3}$ are the area coordinates subjected to the
constraint $\lambda_{1} + \lambda_{2} + \lambda_{3} = 1$ and $0 \leq \lambda_{1},\lambda_{2},
\lambda_{3}\leq 1$ for points inside the triangle.

\section*{Problem 3:}
The triangle lies in the region above the lines $L1,L3$ and below $L2$. Converting it to the form
$Bx\leq d$ we have
\begin{align*}
\left\{\begin{array}{r l}
-x_1-x_2  & \leq -1\\
2x_1+4x_2 & \leq 3\\
x_1-x_2   & \leq 1
\end{array} \right. \implies
B = \left(\begin{array}{r l}
-1  & -1\\
2 & 4 \\
1 & -1
\end{array} \right),
d = \left(\begin{array}{c}
-1\\
3\\
1
\end{array} \right)
\end{align*}

\section*{Problem 4:}
The explicit form of $q(x)=\|Ax\|^2_2=(x_1+x_2)^2+(2x_1+4x_2)^2+(x_1-x_2)^2 $ or \\
$q(x)=2(x_1^2+x_2^2)+4(x_1+2x_2)^2=6x_1^2+18x_2^2+16x_1x_2$. For the eigenvalues of $A^TA$ we have:
\begin{align*}
|A^TA-\lambda I| &= \left|\begin{array}{c c}
6-\lambda & 8\\
8 & 18-\lambda\\
\end{array} \right|
= (6-\lambda )(18-\lambda )-64=\lambda^2-24\lambda-44=0\\
\lambda_1=2&,\lambda_2=22 \quad \text{are the eigenvalues}
\end{align*}

For $ q(x)\geq c\|x\|^2_2 \forall x \in \mathbb{R}^2$ any $c$ can be used when $x_1=x_2=0$. When
$x_1 \neq 0, x_2\neq 0$:
\begin{align*}
c&\leq \frac{q(x)}{\|x\|_2^2}\\
 &= \inf_{\textbf{x}\in \mathbb{R}^2, \textbf{x}\neq 0} \frac{2(x_1^2+x_2^2)+4(x_1+2x_2)^2}{(x_1^2+x_2^2)}
  =2+\inf_{\textbf{x}\in \mathbb{R}^2, \textbf{x}\neq 0}\frac{4(x_1+2x_2)^2}{(x_1^2+x_2^2)}
\end{align*}
or $c=2$ is the largest value when $\textbf{x}\neq 0$

\section*{Problem 5:}
The least squares system of equations associated with this system is minimizing
$f(x):=\|Ax-b\|_2^2$ and from the sufficiency conditions in HW\#3 we have:
$\nabla f(\textbf{x}^*)=0$ or $A^T(A\textbf{x}^*-b)=0$ or $A^TA\textbf{x}^*=A^Tb$.
Using the results in problem 1 we get the linear system: $6x_1^*+8x_2^*=8$ and $8x_1^*+18x_2^*=12$.
Solving it we get: $x_1^*=(\frac{12}{11})$ and $x_2^*=(\frac{2}{11})$. This lies inside the
triangle as marked by the red x in the figure for problem 1.
\begin{align*}
B\textbf{x}^* &= \left(\begin{array}{c}
-1.2727 \\
2.9091 \\
0.9091
\end{array} \right)\leq
d = \left(\begin{array}{c}
-1\\
3\\
1
\end{array} \right)
\end{align*}

\section*{Problem 6:}
\begin{align*}
AA^T=\left(\begin{array}{c c c}
2  &   6  &   0 \\
6  &  20  &  -2 \\
0  &  -2  &   2
\end{array}\right)
\end{align*}
From problem 4 we have the eigenvalues of $AA^T$ are 0, 2 and 22. So, the sigular values of A are 
the squareroots of these eigenvalues $0,\sqrt{2}$ and $\sqrt{22}$. The singular values of $A^T$ are
the square roots of the eigenvalues $A^TA$ which are $\sqrt{2}$ and $\sqrt{22}$.

\section*{Problem 7:}
The changed least squares system of equations associated with this system is minimizing
$\hat{f}(x):=\|C\textbf{x}-e\|_2^2$ with:
\begin{align*}
C &= \left(\begin{array}{c c}
1 & 1\\
2 & 2\\
1 & -1
\end{array} \right),
e = \left(\begin{array}{c c}
1\\
1.5\\
1
\end{array} \right)
\end{align*}
and this again satisfies $C^TC\hat{\textbf{x}}^*=C^Te$. Solving the linear system with
\begin{align*}
C^TC &= \left(\begin{array}{c c}
3 & 2\\
2 & 6
\end{array} \right),
C^Te = \left(\begin{array}{c}
3.5\\
3
\end{array} \right)
\end{align*}
we get $\hat{x_1}^*=\frac{15}{14}$ and $\hat{x_2}^*=\frac{1}{7}$, here
\begin{align*}
B\hat{\textbf{x}}^* &= \left(\begin{array}{c}
-1.2143 \\
2.7143 \\
0.9286 
\end{array} \right)\leq
d = \left(\begin{array}{c}
-1\\
3\\
1
\end{array} \right)
\end{align*}
Hence it is still inside the triangle as shown by green x in figure for problem 1.

This solution is different to the solution in problem 5 because we are minimizing two different
objectives:
\begin{align*}
f(x)&=\|Ax-b\|^2_2=(x_1+x_2-1)^2+(2x_1+4x_2-3)^2+(x_1-x_2-1)^2 \\
\hat{f}(x)&=\|C{x}-e\|^2_2=(x_1+x_2-1)^2+ \left(\frac{1}{2}(2x_1+4x_2-3)\right)^2+(x_1-x_2-1)^2 
\end{align*}
This changes the residuals in the solution of the linear equations as seen above and hence we have
different points as solutions.

\end{document}
